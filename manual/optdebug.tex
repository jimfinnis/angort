\section{Optimisation and debugging}
Angort does not have an optimising compiler: generally, tokens are converted
directly into instructions. It is theoretically possible to write an
optimiser but this has not yet been done. However, some tricks are available
to assist in writing fast code:
\begin{itemize}
\item Use constant expressions rather than ``folding'' constants by
hand (see below).
\item Use the stack (or locals) to store common subexpressions.
\item Use debug trace mode, print messages or breakpoints
to determine which instructions are run most often.
\item Above all, write the code which must run the fastest as native C++.
\end{itemize}
While this last point may seem to defeat the idea of a new language, it
should be remembered that Angort is not intended as a high-performance
language like C++.

\subsection{Constant expressions}
Constant expressions allow the compiler to compile a section of code,
run it, and insert an instruction which will stack the value it produces.
They are delimited by angle brackets, $<<$ and $>>$.
They are useful to ``fold'' long constant expressions into a single value,
which could be a list, hash or closure as well as a scalar or string.
For example, we could write the following code:
\begin{lstlisting}
:foo
    0 1000 range each {
        ["a","b","c","d","e"] each {
          i j dosomething
        }
    };
\end{lstlisting}
The problem here is that the inner list of letters is rebuilt
each time we go around the loop, with the cost of creating the list
and stacking and appending each item to it. If we ``disassemble'' this function
into internal Angort instructions
(which can be done with \texttt{"foo" disasm}, see below) we see this in the generated
code:
\begin{v}
     0x1819ec0 [foo.ang:4] : 0000 : litint (2) (0)
     0x1819ec0 [foo.ang:4] : 0001 : litint (2) (1000)
     0x1819ec0 [foo.ang:4] : 0002 : func (4)  (std$range)
     0x1819ec0 [foo.ang:4] : 0003 : iterstart (44) 
     0x1819ec0 [foo.ang:4] : 0004 : iterlvifdone (43) (offset 19)
     0x1819ec0 [foo.ang:5] : 0005 : newlist (47) 
     0x1819ec0 [foo.ang:5] : 0006 : litstring (18) (a)
     0x1819ec0 [foo.ang:5] : 0007 : appendlist (48) 
     0x1819ec0 [foo.ang:5] : 0008 : litstring (18) (b)
     0x1819ec0 [foo.ang:5] : 0009 : appendlist (48) 
     0x1819ec0 [foo.ang:5] : 0010 : litstring (18) (c)
     0x1819ec0 [foo.ang:5] : 0011 : appendlist (48) 
     0x1819ec0 [foo.ang:5] : 0012 : litstring (18) (d)
     0x1819ec0 [foo.ang:5] : 0013 : appendlist (48) 
     0x1819ec0 [foo.ang:5] : 0014 : litstring (18) (e)
     0x1819ec0 [foo.ang:5] : 0015 : appendlist (48) 
     0x1819ec0 [foo.ang:5] : 0016 : iterstart (44) 
     0x1819ec0 [foo.ang:5] : 0017 : iterlvifdone (43) (offset 5)
     0x1819ec0 [foo.ang:6] : 0018 : func (4)  (std$i)
     0x1819ec0 [foo.ang:6] : 0019 : func (4)  (std$j)
     0x1819ec0 [foo.ang:6] : 0020 : globaldo (5) (user$something)
     0x1819ec0 [foo.ang:7] : 0021 : jump (15) (offset -4)
     0x1819ec0 [foo.ang:8] : 0022 : jump (15) (offset -18)
     0x1819ec0 [foo.ang:9] : 0023 : end (1) 
\end{v}
Note the \texttt{newlist} followed by \texttt{litstring/appendlist} pairs.
These take time, and this gets much worse with a longer list. It is possible
to deal with this by generating the list and putting it in a variable:
\begin{lstlisting}
:foo |:lst|
    ["a","b","c","d","e"] !lst
    0 1000 range each {
        ?lst each {
          i j dosomething
        }
    };
\end{lstlisting}
which generates the code:
\begin{lstlisting}
     0x1da2ec0 [foo.ang:4] : 0000 : newlist (47) 
     0x1da2ec0 [foo.ang:4] : 0001 : litstring (18) (a)
     0x1da2ec0 [foo.ang:4] : 0002 : appendlist (48) 
     0x1da2ec0 [foo.ang:4] : 0003 : litstring (18) (b)
     0x1da2ec0 [foo.ang:4] : 0004 : appendlist (48) 
     0x1da2ec0 [foo.ang:4] : 0005 : litstring (18) (c)
     0x1da2ec0 [foo.ang:4] : 0006 : appendlist (48) 
     0x1da2ec0 [foo.ang:4] : 0007 : litstring (18) (d)
     0x1da2ec0 [foo.ang:4] : 0008 : appendlist (48) 
     0x1da2ec0 [foo.ang:4] : 0009 : litstring (18) (e)
     0x1da2ec0 [foo.ang:4] : 0010 : appendlist (48) 
     0x1da2ec0 [foo.ang:4] : 0011 : localset (8) (idx 0)
     0x1da2ec0 [foo.ang:5] : 0012 : litint (2) 
     0x1da2ec0 [foo.ang:5] : 0013 : litint (2) 
     0x1da2ec0 [foo.ang:5] : 0014 : func (4)  (std$range)
     0x1da2ec0 [foo.ang:5] : 0015 : iterstart (44) 
     0x1da2ec0 [foo.ang:5] : 0016 : iterlvifdone (43) (offset 9)
     0x1da2ec0 [foo.ang:6] : 0017 : localget (7) (idx 0)
     0x1da2ec0 [foo.ang:6] : 0018 : iterstart (44) 
     0x1da2ec0 [foo.ang:6] : 0019 : iterlvifdone (43) (offset 5)
     0x1da2ec0 [foo.ang:7] : 0020 : func (4)  (std$i)
     0x1da2ec0 [foo.ang:7] : 0021 : func (4)  (std$j)
     0x1da2ec0 [foo.ang:7] : 0022 : globaldo (5) (idx dosomething)
     0x1da2ec0 [foo.ang:8] : 0023 : jump (15) (offset -4)
     0x1da2ec0 [foo.ang:9] : 0024 : jump (15) (offset -8)
     0x1da2ec0 [foo.ang:10] : 0025 : end (1) 
\end{lstlisting}
This is much better: the list is generated and stored before the loop
starts at offset 0015, so it is only generated once per function call.
However, we can improve on this for both legibility and speed using
a constant expression:
\begin{lstlisting}
:foo
    0 1000 range each {
        <<["a","b","c","d","e"]>> each {
          i j dosomething
        }
    }
;
\end{lstlisting}
This is the same as the original code, but with angle brackets around
the code which makes the list. Here's the disassembly:
\begin{lstlisting}
     0x13a1410 [foo.ang:4] : 0000 : litint (2) 
     0x13a1410 [foo.ang:4] : 0001 : litint (2) 
     0x13a1410 [foo.ang:4] : 0002 : func (4)  (std$range)
     0x13a1410 [foo.ang:4] : 0003 : iterstart (44) 
     0x13a1410 [foo.ang:4] : 0004 : iterlvifdone (43) (offset 9)
     0x13a1410 [foo.ang:5] : 0005 : constexpr (61) 
     0x13a1410 [foo.ang:5] : 0006 : iterstart (44) 
     0x13a1410 [foo.ang:5] : 0007 : iterlvifdone (43) (offset 5)
     0x13a1410 [foo.ang:6] : 0008 : func (4)  (std$i)
     0x13a1410 [foo.ang:6] : 0009 : func (4)  (std$j)
     0x13a1410 [foo.ang:6] : 0010 : globaldo (5) (idx dosomething)
     0x13a1410 [foo.ang:7] : 0011 : jump (15) (offset -4)
     0x13a1410 [foo.ang:8] : 0012 : jump (15) (offset -8)
     0x13a1410 [foo.ang:9] : 0013 : end (1) 
\end{lstlisting}
Here, we are no longer generating the list at all -- it has already
been generated \emph{at compile time.} Instead, we compile a single
\texttt{constexpr} instruction to push the list onto the stack.
Here's what happens when we encounter a $<<$ during compilation:
\begin{enumerate}
\item Store the current compile context and begin a new one, as if we 
were compiling a new function.
\item Compile code until we reach a $>>$ sequence.
\item Run the code we just compiled.
\item Pop a value from the top of the stack.
\item Restore the original compile context.
\item Compile a \texttt{constexpr} instruction containing the value
we just popped off the stack.
\end{enumerate}
The most important thing to realise about constant expressions is that
they break lexical scope: they cannot access any of the variables of
the functions in which they are embedded. This is because they are not
actually part of those functions: they are a completely separate piece of
code run at compile time, long before any containing function ever
runs. Thus the following code is invalid:
\begin{lstlisting}
:foo |:a|
    2!a
    << 2 ?a * >> # invalid
;
\end{lstlisting}
because the local variable \texttt{a} is not in the scope of the constant
expression. Although the function contains the expression lexically, it
is not a true part of the function. Globals (provided they are created
before the constant expression) are fine, because Angort compiles and
runs each line at a time (outside functions).
\begin{lstlisting}
2 !A
:foo
    <<3 ?A *>>. # valid - A exists when this compiles
;   
\end{lstlisting}

\subsection{Using constant expressions in closures}
We often write functions to create other functions. For example,
consider an exponential smoothing function of the form
\[
y_t = \beta x_t + (1-\beta)y_{t-1}
\]
We can write a function to generate a smoother:
\begin{lstlisting}
:mksmoother |beta:y|
0!y
    (|x:| ?beta ?x * 1.0 ?beta - ?y * + dup !y)
;   
\end{lstlisting}
which we can then use:
\begin{lstlisting}
0.1 mksmoother !F
{ read tofloat ?F@ . }
\end{lstlisting}
but this requires us to create a variable for each smoother,
which could become awkward if we need to create a large number of them
and would separate their creation from where they are used. Let's imagine
we are dealing with many streams of interleaved input, each of which
should be smoothed separately and perhaps with a different constant:
\begin{lstlisting}
0.1 mksmoother !F1
0.12 mksmoother !F2
0.3 mksmoother !F3
0.1 mksmoother !F4
(
    { 
        read tofloat ?F1@.
        read tofloat ?F2@.
        read tofloat ?F3@.
        read tofloat ?F4@.
    }
)@
\end{lstlisting}
Instead, we could create the smoothers using constant expressions:
\begin{lstlisting}
(
    { 
        read tofloat <<0.1 mksmoother>>@.
        read tofloat <<0.12 mksmoother>>@.
        read tofloat <<0.3 mksmoother>>@.
        read tofloat <<0.1 mksmoother>>@.
    }
)@
\end{lstlisting}
which is much neater. Constant expressions have many creative uses!


\subsection{Tracing and disassembling}
\todo{tracing}


\subsection{The debugger}
\todo{debugger}
