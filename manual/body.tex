\section{Introduction}
Angort is a stack-based concatenative programming language with some
functional features. The language has grown from a simple Forth-like
core over time, and has been used primarily for robot control on
an ExoMars rover locomotion prototype.

This is an extremely brief introduction to the language. It may
be useful for readers unfamiliar with this style of programming
to look into Forth, which is an older, more primitive (but faster and
smaller) language from which much of the syntax of Angort was
borrowed.

It combines the power and ease of a modern dynamic language with
the convenience of a Forth-like language for controlling robots in
real time. For example, on our rover we have the following definitions
in the startup file:

\begin{lstlisting}[style=ang]
# define a constant "wheels" holding a range from 1-6 inclusive

range 1 7 const wheels

# define a new word "d" with a single parameter "speed"

:d [speed:]

    # set a help text for this word

    :"(speed --) set speed of all drive motors"
    
    # for each wheel, set the required speed to the value
    # of the parameter
    
    wheels each {
        ?speed i!drive
    }
;

# slightly more complex word for steering

:t [angle:]
    :"(angle --) turn front wheels one way, back wheels opposite way"
    
    ?angle dup 1!steer 2!steer
    0 0 3!steer 4!steer
    ?angle neg dup 5!steer 6!steer
;

# define a word to stop the rover by setting all speeds to zero

:s 0 d;

\end{lstlisting}
Once these words are defined we can steer the robot in real time with
commands like:
\begin{v}
2500 d
30 t
s
\end{v}
These will set the rover speed to 2500, turn it to 30 degrees, and stop
it respectively. We can also directly type things like:
\begin{v}
wheels each { i dactual .}
\end{v}
which will print the actual speeds of all the wheels. As can be seen,
it is a very terse language.

In the examples given so far, words such as \texttt{dactual},
\texttt{!drive} and \texttt{?drive} are links to native C++ code: it
is very easy to interface Angort with C++.

\section{Basic concepts}
Angort, like Forth (with which it shares many conceptual elements)
is a stack-based language. It consists of sequences of ``words'',
which can be constants (such as strings or numbers), or functions.
All words operate on the stack, or manipulate the internal state,
in sequence --- there is very little ``syntax'' as such.

Because of this, Angort is a \emph{reverse Polish} language: functions
(including operations like $+$ and $-$) precede their operands. For
example, to print $(123+231)*6$ we would write
\begin{v}
123 231 + 6 * .
\end{v}
This would perform the following actions:
\begin{itemize}
\item push 123 onto the stack
\item push 231 onto the stack
\item pop the top two values off the stack, add them, and push the sum (giving $123+231$)
\item push 6 onto the stack (so the stack now has $123+231$ and $6$)
\item pop the top two values off the stack, add them, and push the product
\item pop the top value and print it.
\end{itemize}
A few words don't work like this. For example, the \texttt{const} keyword
pops a value off the stack, reads the next token in the input stream and
creates a constant of that name. 


