



\section{Lists}
\label{lists}
\index{lists}\indw{[|textbf}\indw{]|textbf}
Angort lists are actually array based, with the array resizing automatically
once certain thresholds are passed (similar to a C++ vector or Java ArrayList).
A list is created by enclosing Angort expressions in square
brackets and separating them by commas:
\begin{v}
[]              # the empty list
[1,2,3]         # a list of three integers
["foo","bar",1] # two strings and an integer
\end{v}
As noted above, lists can be iterated over. Lists can also contain lists,
and can be stored in variables ---  more precisely, references to lists can
be stored in variables:
\begin{v}
1|0> [1,2] !A       # create a list and store it in global A
2|0> ?A!B           # copy A to B
2|0> 3 ?A push      # append an item to A
2|0> ?A each {i.}   # print A
1
2
3
1|0> ?B each {i.}   # print B - it also has the extra item!
1
2
3
\end{v}
Note that the list in $B$ has also changed --- it is the same list,
just a different reference.
The following are some words which act on lists, with their stack pictures:
\indw{[|textbf}\indw{]|textbf}\indw{,|textbf}\indw{get}
\indw{set}\indw{remove}\indw{shift}\indw{unshift}
\indw{push}\indw{pop}\indw{in}
\begin{center}
\begin{tabular}{|l|l|p{4in}|}\hline
\textbf{name} & \textbf{stack picture} & \textbf{side-effects and notes}\\ \hline
[    & (-- list)    & creates a new list\\
,    & (list item -- list) & appends an item to the list\\
]    & (list item -- list) & appends an item to the list\\
get & (n list -- item) & get the nth item from the list\\
set & (item n list --) & set the nth item in the list\\
remove & (n list -- item) & remove and return the nth item\\
shift & (list -- item) & remove and return the first item\\
unshift & (item list --) & prepend an item\\
pop & (list -- item) & remove and return the last item\\
push & (item list --) & append an item\\
in & (item iter --) & return true if an item is in a list, integer range or hash keys\\
\hline
\end{tabular}
\end{center}
Note that the literal notation for lists --- the square brackets and the
comma --- fall naturally out of the definition of the words\footnote{There
is an exception: if the tokeniser finds the sequence \texttt{[]} it discards
the second bracket, allowing us to notate the empty list in a natural way.}.
The comma
is a useful word, acting as a way to add items to a list on top of the stack
without popping that list. This means we can write code to do a list copy:
\begin{lstlisting}
:copylist |list:|
    []              # stack an empty list
    ?list each {    # iterate over the list passed in
        i ,         # for each item, add it to the list on the stack
    }
;                   # finish and return the list on the stack          
\end{lstlisting}
This syntax allows us to create a form of list comprehension:
\begin{lstlisting}
[] [1,2,3,4] each {i dup * dup 2 % if , else drop then}
\end{lstlisting}
will create the list $[1,9]$: the squares of the numbers but only 
if they are odd. It does this by squaring the number, duplicating and
testing to see if it is nonzero modulo 2, and if so appending it to the list,
otherwise dropping it. 
In reality, we would probably use the \texttt{map} and
\texttt{filter} words with anonymous functions:
\begin{lstlisting}
[1,2,3,4] (dup *) map (2 %) filter
\end{lstlisting}


\subsection{slice -- getting parts of lists or strings}
The \texttt{slice} word can be used to extract part of a list or string,
returning a smaller list or string. The semantics are identical for both
strings and lists.

At version 3, Angort is in a transition
phase with two implementations of \texttt{slice}, each with entirely different
semantics! This is due to the older code being both conceptually wrong
and buggy, but still in use. 

In the \texttt{future} namespace (see , \texttt{slice} has similar semantics
to the Python array slice operator. In the \texttt{deprecated} namespace,
a crude slice operator is supported which is briefly documented below
(and has bugs with negative indices). The default implementation of \texttt{slice} will
throw an exception -- either \texttt{future} or \texttt{deprecated} must
be imported, for example with
\begin{lstlisting}
`future nspace [`slice] import
\end{lstlisting}


\subsubsection{The \texttt{future} implementation}
This will become the default implementation in Angort 4.x.x, when
existing code has been corrected. This implementation uses similar
semantics to Python:
\begin{itemize}
\item The stack picture is \texttt{(list/string start end -- out)} where both \texttt{start} and \texttt{end} 
are zero-based indices. The slice includes the start and excludes the end,
so 
\begin{lstlisting}
"foo" 0 2 slice.
\end{lstlisting}
will print
\begin{v}
fo
\end{v}
\textbf{A zero end index maps to the length of the sequence.} This means that
\begin{lstlisting}
"fooble" 1 0 slice
\end{lstlisting}
will print from character 1 to the end of the string:
\begin{v}
ooble
\end{v}

\item Negative indices (or a zero end index as mentioned above) count from the end, so -1 is the last element, -2
is the penultimate element etc. Thus, 
\begin{lstlisting}
"fooble" -4 -1 slice.
\end{lstlisting}
will print from the fourth character from the end to just before the last character:
\begin{v}
obl
\end{v}
\end{itemize}
    




\subsubsection{The \texttt{deprecated} implementation}
This is the default implementation for version 2.x.x and lower.
Here, the stack picture is \texttt{(list/string start len -- out)}: the arguments
are the zero-based start index and the number of elements in the slice.
If the length is negative the slice is to the end of the string or list.

The start index may be also be negative, indicating distance from the
end, but this is bugged for lists.



\section{Symbols}
\index{symbols}\indw{`}
Symbols are single-word strings which are stored in an optimised form for easy
and quick comparison (they are turned into unique integers
internally). They are specified by using a backtick (\texttt{`}) in front of
the word. They're most useful as keys in hashes, covered in the next section.
Examples are \texttt{`foo}, \texttt{`bar}. Symbols can be used
most places where strings are used, but cannot be iterated or sliced.

\section{Hashes}
\label{hashes}
\index{hashes}
Hashes are like Python dictionaries: they allow data be stored using keys of any hashable type (see table in Sec.~\ref{types}).
Hashes are created
using a similar syntax to the list initialiser, but with a \texttt{\%} after
the closing brace and data in key,value pairs\footnote{This is a somewhat
awkward syntax, but all the other bracket types were used elsewhere.}:
\indw{[\%]}
\begin{v}
[%]
\end{v}
creates an empty hash, and 
\begin{v}
[%
    `foo "the foo string",
    `bar "the bar string"
]
\end{v}
creates a hash with two entries, both of which are keyed by symbols (although
the keys in a hash can be of different types). We can add values to the hash
using \texttt{set} which has the picture \texttt{(val key hash --)}:

\begin{v}
[%]!H                           # create the empty hash
"the foo string" `foo ?H set    # add a string with the key `foo
"the bar string" `bar ?H set    # add a string with the key `bar
\end{v}
and read values using \texttt{get} which has the picture \texttt{(key hash -- val)}:
\begin{v}
2|0> `foo ?H get
2|1> .
the foo string
\end{v}
\indw{ival}
We can also iterate over the hash's keys and values:
\begin{lstlisting}
:dumphash |h:|
    ?h each {                   # iterate over the hash's keys
        i p                     # print key without trailing new line
        ":   " p                # print colon and spaces
        ival                    # get the value of the key in the hash
        .                       # and print it
    }
;
\end{lstlisting}
If we run this on the hash $H$ defined above, we get
\begin{v}
2|0> ?H dumphash
foo:   the foo string
bar:   the bar string
\end{v}
\indw{ival}\indw{jval}\indw{kval}
We can nest iterator loops for hashes too, as we did with lists.
Just as \texttt{i}, \texttt{j} and \texttt{k} get the inner, next outer
and next outer loop keys (see Sec.~\ref{nestit}), \texttt{ival},\texttt{jval} and \texttt{kval} get
the inner, next outer and next outer loop values with hashes.


\subsection{Shortcut symbol get and set}
\indw{`}\index{hashes!shortcut set/get}
There is ``syntactic sugar'' for retrieving the value of a key in
a hash where the key is a literal symbol. 
The syntax is \verb+?`key+ , meaning
``get the value for \emph{key} in the hash on the stack,'' and so has the stack picture \texttt{(hash -- value)}.
Therefore, instead of using
\begin{v}
`foo ?H get
\end{v}
we can use
\begin{v}
?H?`foo
\end{v}
This has been added because this is by far the most common use-case.
We also have the same ability to set a value in a hash with a literal
symbol, using the syntax \verb+!`key+. This has the stack picture \texttt{(value hash --)}.
Thus we can do
\begin{v}
96 ?H!`temperature
\end{v}
instead of
\begin{v}
96 `temperature ?H set
\end{v}
\subsection{Words for hashes}
Hashes can also use many of the same words as lists:
\indw{[\%}\indw{,}\indw{set}\indw{get}\indw{remove}\indw{in}
\begin{center}
\begin{tabular}{|l|l|p{4in}|}\hline
\textbf{name} & \textbf{stack picture} & \textbf{side-effects and notes}\\ \hline
[\%    & (-- hash)    & creates a new hash\\
,    & (hash key value -- hash) & adds a value to the hash\\
]    & (hash key value -- hash) & adds a value to the hash\\
get & (key hash -- value) & get a value from the hash, or \texttt{none} if it is not present\\
set & (value key hash --) & set a value in the hash\\
remove & (key hash -- value) & remove and return a value by key\\
in & (key hash --) & return true if a key is in the hash\\
\hline
\end{tabular}
\end{center}
Note that the comma and close bracket words examine the stack to
determine if they are working on a list or a hash.

\subsection{Hash to string function}
\indw{toString}
By default, printing a hash --- or converting it to a string any other way ---
will just print the default string. This tells you it's a hash and gives
its address in memory:
\begin{v}
1|0 > [%] .
<TYPE hash:0x16a4840>
2|0 > 
\end{v}
However, if we define a hash member called \verb+toString+ (where this
key is a symbol) which is a function, then that function will be called
to generate a string. This is useful in many cases where hashes are used
as data structures\footnote{However, this can sometimes go horribly
wrong, particularly where debugging is involved. Use with care.}.

\section{Garbage collection}
\index{garbage collection}
Garbage collection is done automatically --- up to a point. Specifically,
the system does reference-counted garbage collection continuously,
which works well in most cases. However, it is possible to
create cycles:
\begin{v}
    [] !A              # make a list called A
    [?A] !B            # make a list called B, referencing A
    ?B ?A push         # add a reference to B in A
\end{v}
Now there are two objects referencing each other --- a cycle. This can
happen in lists, hashes and closures. Reference-counted garbage
collection will never delete these. Therefore it may be necessary
in programs with a complex structure to call the full garbage collector
occasionally.

This is done periodically, by default every 100000 instructions or so.
This interval can be changed by writing to the \verb+autogc+ property
with a new interval:
\indw{autogc}
\begin{v}
1000 !autogc
\end{v}
It can also be disabled entirely by setting \texttt{autogc} to a negative
value.
A full garbage collect can be done manually by the word
\indw{gc}
\begin{v}
    gc
\end{v}
Incidentally, this is
the same style of garbage collection used by Python.



\section{Functional programming}
\label{functional}
\index{functional programming}\index{anonymous functions}
\indw{(}\indw{)}
Anonymous functions are defined with brackets, which will push an object
representing that function (and any closure created) onto the stack. This can
then be called with \texttt{call}  (which can be abbreviated to ``\texttt{@}'')
\indw{call}\indw{"@}
Such functions may have parameters and local variables.
For example, this is a function to run a function over a range of numbers,
printing the result:
\begin{lstlisting}
:over1to10 |func:|
    1 10 range each { i ?func@ . } ;
\end{lstlisting}
With this defined, we can now use it to show the squares of those
numbers:        
\begin{v}
    (|x:| ?x dup *) over1to10
\end{v}
or more simply
\begin{v}
    (dup *) over1to10
\end{v}

\subsection{Recursion}
\todo{This doesn't really belong here; should it go into a putative
``debugging and optimisation'' section?}
Recursion is normally achieved by calling the function by name within
its own definition:
\begin{lstlisting}
:factorial |x:|
    ?x 1 = if 1 else ?x ?x 1 - factorial * then;
\end{lstlisting}
This is not possible in an anonymous function. To achieve recursion
in anonymous functions, use the
\texttt{recurse} keyword to make the recursive call:
\begin{lstlisting}
1 10 range (|x:| ?x 1 = if 1 else ?x ?x 1 - recurse * then) map
"," intercalate.
\end{lstlisting}
will print the factorials of the first 10 natural numbers. See below
for how \texttt{map} performs a function on each member of an iterable
to produce a list, and for how \texttt{intercalate} builds a string
out of a list by joining string representations of its members with
a separator.

The keyword \texttt{self} is occasionally useful: rather than calling
the containing function recursively, it stacks a reference to the
function itself. Thus:
\begin{lstlisting}
:foo inc self !LastFuncCalled;
4 foo.
4 ?LastFuncCalled@.
\end{lstlisting}
creates a function \texttt{foo} which, when called, increments the
value on the stack and also stores a reference to \texttt{foo} in
the global \texttt{LastFuncCalled}. We then call \texttt{foo},
and then call whatever is stored in \texttt{LastFuncCalled} -- which will
be \texttt{foo} again.
\subsubsection{A warning}
Angort has a limited return stack size of 256 frames, and because of the nature
of the language there is no tail call optimisation. Recursive algorithms
may therefore run out of stack. Also, Angort may not be suitable for expressing
very complex recursive functions. Consider for example the quicksort algorithm:
this can be implemented as
\begin{lstlisting}
:qs |l:p| 
    ?l len 1 <= if 
        ?l 
    else 
        ?l pop !p 
        ?l (?p <) filter qs 
        [?p] + 
        ?l (?p >=) filter qs + 
    then;
\end{lstlisting}
but running it on a large number of items will be very slow. Try
\begin{lstlisting}
[] 0 100000 each {rand 200000 %,} qs
\end{lstlisting}
This generates a list of 100000 integers in the range $[0,199999]$ and
sorts them. On my laptop it takes about 7 seconds -- a long time for
such a simple task.
This is because the algorithm recurses deeply and widely,
and each recursion constructs 
three lists (a new single-item list for the pivot and two lists
using a filter) and pastes them together, constructing a temporary 
list on the way. This is very inefficient.
In contrast, using the built in \texttt{sort} word takes only 0.25s:
\begin{lstlisting}
[] 0 100000 each {rand 200000 %,} sort
\end{lstlisting}
This is still slow, because the internal comparison operator
needs to perform typechecking on each pair of elements it compares
so that integers and floats in the same list will be compared
correctly. However, it is a big improvement because the \texttt{libc} 
\texttt{qsort} function is being used.


\subsection{Words for dealing with functions}
\indw{map}\indw{reduce}\indw{filter}\indw{zipWith}
\begin{center}
\begin{tabular}{|l|l|p{4in}|}\hline
\textbf{name} & \textbf{stack picture} & \textbf{side-effects and notes}\\ \hline
map &(iter func -- list) & apply a function to an iterable, giving a list\\
reduce & (start iter func -- result) & set an internal value (the accumulator) to "start", then iterate, applying the function (which must take two arguments) to the accumulator and the iterator's value, setting the accumulator to this new value before moving on.\\
filter & (iter func -- list) & filter an iterable with a boolean function\\
filter2 & (iter func -- falselist truelist) & filter an iterable with a boolean function, placing true elements and false elements in separate lists\\
zipWith & (iter iter func -- list) & iterate over both iterables, combining
the elements with a binary function and putting the results into a list\\
all & (iterable func -- bool) & true if the function returns true for all
members of the iterable.\\
any & (iterable func -- bool) & true if the function returns true for any
members of the iterable.\\
\hline
\end{tabular}
\end{center}
\clearpage
We can now list all the squares of a range of numbers:
\begin{lstlisting}
0 100 range (dup *) map each {i.}
\end{lstlisting}

We can also write a function to sum the values in an iterable using
\texttt{reduce}:
\begin{lstlisting}
:sum |itr:|
    0           # the accumulator value starts at zero
    ?itr        # this is the iterable
    (+)         # and this is the function 
    reduce      # repeatedly add each item to the accumulator,
                # setting the accumulator to the result. When
                # finished, return the accumulator.
;
\end{lstlisting}



\subsection{Closures}\index{closures}
Anonymous functions can refer to variables in their enclosing function or
function, in which case a closure is created to store the value when the enclosing
function exits. This closure is mutable - its value can be changed by the
anonymous function. For example, consider the following function:

\begin{lstlisting}
:mkcounter |:x|     # declare a local variable x
    0!x             # set it to zero
    (               # create a function
        ?x dup .    # which prints the local
        1+ !x       # and increments it
    )
;
\end{lstlisting}
This returns an anonymous function which refers to the local variable
inside the function which created it. We can run this function
and store the returned function in a global:

\begin{v}
mkcounter !F    # run it and store the returned function+closure
\end{v}
If we now run
\begin{v}
    ?F call
\end{v}
a few times, we will see an incrementing count - the value in the closure
persists and is being incremented. We can call mkcounter several times and
each time we will get a new closure.

\subsubsection{Closures are by reference}
However, all closures are \emph{by reference} --- the child functions
get references to variables closed in the parent function, so a function
which returns several functions will all share the same closure, and
any changes to variables in the closure will be reflected in all the 
other functions. For example:
\begin{lstlisting}
:mklistoffunctions |:x|
    []
    0 10 range each {
        i !x (?x),
    }
;
\end{lstlisting}
looks like it should produce a list of functions, each of which
returns the numbers from 0 to 9. However, all the functions will
return 9 because they all share the same copy of \texttt{x}.
We can get around this by using a \emph{closure factory} function to hold
private copies:
\begin{lstlisting}
:factory |x:| (?x);

:mklistoffunctions |:x|
    []
    0 10 range each {
        i factory,
    }
;

\end{lstlisting}

\subsubsection{Iterators are not stored in closures}
Notice that in the \texttt{mklistoffunctions} example above we did not write
\begin{lstlisting}
:mklistoffunctions
    []
    0 10 range each {
        (i),
    }
;
\end{lstlisting}
and instead used a local variable \emph{x} to store the iterator value. This is because
the values of loop iterators such as \texttt{i} are not true variables, and thus are not
stored in the closure. Iterators must be stored in local variables if they are to be used
in an anonymous function created in their loop's context.

\section{Exceptions}
Exception handling is done with a \texttt{try}/\texttt{catch}/\texttt{endtry} 
construction:
\begin{lstlisting}
try
    ...
catch:symbol1,symbol2...
    ...
endtry
\end{lstlisting}
Upon entry to the catch block, the stack will hold the exception symbol
on the top with extra data (typically a string with more information)
under that. There are quite a few built-in exceptions: they are listed
in \texttt{exceptsymbs.h}. To throw your own exception, use \\
\texttt{throw (data except --)}, e.g.
\begin{lstlisting}
"Can't open file: " ?fn + `badfile throw 
\end{lstlisting}
To catch all exceptions, use the \texttt{catchall} word:
\begin{lstlisting}
try
    ...
catch:symbol1,symbol2...
    ...
catchall
    ...
endtry
\end{lstlisting}





\section{Getting help}
\indw{??}\indw{help}
There are many other functions and operations available in Angort.
These can be listed with the \texttt{list} word, and help can
be obtained on all words with \texttt{??} (except the very low-level words compiled
directly to bytecode, which are all covered above):
\begin{v}
1|0 > ??filter
filter: (iter func -- list) filter an iterable with a boolean function
1|0 > 
\end{v}
If you have loaded a library or package and not imported the functions
into the main namespace (see section~\ref{nameslibsmods}below), you
can use the fully qualified name to get help:
\begin{v}
1|0 > `io library drop
1|0 > ??io$open
io$open: (path mode -- fileobj) open a file, modes same as fopen()
1|0 > 
\end{v}

\section{Namespaces, libraries and packages}
\label{namespaces}
\index{namespaces}
\def\dollarsign{\$}\indw{\dollarsign}
All Angort identifiers (apart from the built-in tokens listed in Table~\ref{tab:builtins})
are stored in a \emph{namespace}. An identifier in a program
can be either \emph{fully qualified},
in which case the namespace for the identifier is explicitly given before
a dollar sign, such as \verb+std$quit+; or \emph{unqualified}, in
which case Angort will scan only the imported namespaces to find it.
Angort starts with a set of default namespaces, which are imported
by default. These include
\begin{itemize}
\item \texttt{coll} for collection functions,
\item \texttt{string} for string functions,
\item \texttt{math} for mathematical functions,
\item \texttt{env} for system environment functions (command line arguments,
environment variables etc.),
\item \texttt{coll} for collection handling functions,
\item \texttt{user} for names defined by the user.
\end{itemize}
Others may be added. Since these namespaces are imported, the user
does not need to enter the fully qualified name. As noted above in Sec.~\ref{futdep},
the \texttt{future} and \texttt{deprecated} namespaces are not imported.
A namespace can be imported by using a command of the form
\begin{lstlisting}
`mynamespace nspace import
\end{lstlisting}
or to only import some names
\begin{lstlisting}
`mynamespace [`name1,`name2 ..] import
\end{lstlisting}
See below for more ways to manipulate namespaces.


\subsection{Packages}
\label{packages}
\indw{package}\indw{require}
\index{packages}
It is possible to define a new namespace using the \texttt{package}
directive, typically done inside a separate file.
Build a package by putting the \texttt{package} directive at the start
of the file along with the package name, e.g.
\begin{v}
package wibble
\end{v}
and include that file with \texttt{require} instead of the usual \texttt{include.} 

This will cause a new namespace to be created where \texttt{package} is 
invoked, and all subsequent definitions until the end of the file will
be put into that namespace. On return from the file, \texttt{require} will
\index{NSID}\index{namespace identifier}\index{namespaces!identifiers}
put a namespace identifer (or NSID) on the stack. All the globals,
constants and functions defined in the package are available by prefixing
the name with the package name and a dollar sign:
\begin{v}
package$name
\end{v}
Note that the package name is that given to the \texttt{package} directive,
not the name of the file! Angort returns to defining things in the \texttt{user} namespace
at the end of the file (but see \texttt{endpackage} below in Sec.~\ref{localpack} if we want to go back to the \texttt{user}
namespace before then).

\subsection{Importing packages into the default space}
\indw{import}
With the NSID returned by \texttt{require} 
n the stack, we can import the namespace --- either
all of it or part of it. The \texttt{import} word takes two forms:
\begin{v}
require "wibble.ang" import
\end{v}
will import all the public names defined in the package, while
\begin{v}
require "wibble.ang" [`foo, `bar] import
\end{v}
will only import the given names --- in this case, \texttt{foo} and \texttt{bar}. If we do not wish
to import the package at all we can do
\begin{v}
require "foo.ang" drop
\end{v}
to discard the NSID.
The namespace identifier can also be used in other ways:
\begin{center}
\begin{tabular}{|l|l|p{3in}|}\hline
\indw{package}\indw{require}\indw{public}\indw{private}
\indw{names}\indw{import}\indw{ispriv}\indw{isconst}\indw{names}
\textbf{name} & \textbf{stack picture} & \textbf{side-effects and notes}\\ \hline
require ``filename'' & (-- nsid) & load a package\\
library & (libname -- nsid) & load a native plugin library (see Sec.~\ref{library})\\
nspace & (name -- nsid) & look up a loaded package or library and return the NSID \\
package packagename & & directive to start a new namespace \\
private & & all subsequent names are not exportable from the namespace\\
public & & all subsequent names are exportable from the namespace (default)\\\hline
import &(nsid --) & import all public definitions from a namespace into the default namespace\\
import &(nsid symbollist --) & import some public definitions from a namespace into the default namespace\\
names & (nsid -- symbollist) & get a list of names in a namespace\\
ispriv & (nsid name -- bool) & return true if a name is private in the namespace\\
isconst & (nsid name -- bool) & return true if a name is constant in the namespace\\
\hline
\end{tabular}
\end{center}

\subsection{Loading and importing libraries}
C++ plugin libraries -- which are dealt with more fully in Sec.~\ref{library} -- are
imported in a similar way using the \texttt{library} function:
\texttt{`io library import} 
will load the IO library and import all its functions. Note the difference, however:
because \texttt{require} is a directive dealt with by the compiler, the name of the package
follows the keyword. The word \texttt{library} is an actual function, which requires a string
on the stack, so the library name (string or symbol) precedes \texttt{library}.
This loads the library and returns the NSID for importing (or dropping).

\subsection{Overriding functions in default namespaces}
Although they are defined as constants, it possible to override the definitions of functions defined
in default imported namespaces
like \texttt{std\$p} and \texttt{std\$quit}. For example the standard function \texttt{p}
can be overridden by forcing a
new global in the \texttt{user} namespace:
\begin{lstlisting}
global p
:p "wibble" .;
\end{lstlisting}
Once this has been done, we now have two \texttt{p} functions: \texttt{std\$p}, which is
the standard definition; and \texttt{user\$p}, which is our new function. Because \texttt{user} is
the ``current'' namespace, i.e. that into which new names are defined, this will be searched
first on compilation, so unqualified \texttt{p} will be resolved as \texttt{user\$p}, our new function.
It is still possible to use \texttt{std\$p} by using the fully qualified name. We could also
define a user function with the same name as a standard function, which uses that standard function,
by using the fully qualified name:
\begin{lstlisting}
global quit
:quit "Angort is quitting now!". std$quit;
\end{lstlisting}
 
\subsection{Local packages}
\label{localpack}
\index{packages!local}
It is sometimes necessary to create a package inside a script which
is not included from another script. One example is where the script
does not terminate, leaving the user at the prompt with some functions defined,
but some functions should be private.

To do this, define the package thus:
\indw{endpackage}
\begin{v}
package somename
private
...private functions...
public
...public functions...
endpackage import
\end{v}
The \texttt{endpackage} word does the same as the return from a 
\texttt{require} normally does --- close off the package and
stack the package ID.

\subsection{The ``deprecated'' and ``future'' namespaces}
\label{futdep}
Angort is a changing language, and to permit this some functions are
occasionally moved into the ``deprecated'' namespace when they are obsolete.
Other functions are added to the ``future'' namespace when they are new
and break back-compatibility. 
While both these namespaces are loaded, none of their symbols are imported. 
Typically, a function might have an
old version in the deprecated space and a new version in the future space,
while the default namespace contains a placeholder which gives details
of the differences between the versions and throws an exception.
it is up to the user to explicitly import either the deprecated or
future version.

To do this, we need to import symbols from a namespace which is already
loaded, which we can do using the \texttt{nspace} function; and then
use \texttt{import}. For example, if we wish to use \texttt{foo} and
\texttt{bar} from the future namespace but \texttt{fish} from the deprecated
namespace, we could use
\begin{lstlisting}
`future nspace [`foo,`bar] import
`deprecated nspace [`fish] import
\end{lstlisting}



\section{The true nature of function definitions}
\label{globdetails}
Words are actually global variables bound to anonymous functions.
Given that anonymous functions are written as blocks of Angort
in brackets (see below), then
\begin{lstlisting}
:square |x:| ?x ?x *;
\end{lstlisting}
could also be written as
\begin{lstlisting}
global square
(|x:| ?x ?x *) !square
\end{lstlisting}
with exactly the same end result. Referring to a global by using the \texttt{?} sigil
will simply stack its value, whereas referring to it without the sigil
will check if it is holds a code block or closure and run it if so, otherwise
stack the value. This is useful in functional programming.

\subsection{Sealed functions}
\index{const functions}
Combined with constants, this allows for ``sealed'' function definitions. 
In general all functions can be redefined, but a function defined thus:
\begin{lstlisting}
(|x:| ?x ?x *) const square
\end{lstlisting}
cannot be. This can be useful in package development.

This is also useful when you want to import only certain names from
a plugin library (Section~\ref{library}). Consider the \texttt{vector2d} library, which contains
the functions \texttt{vector2d\$x} and \texttt{vector2d\$y}. These
are cumbersome, so we could import just these in their short forms
as
\begin{lstlisting}
?vector2d$x const x
?vector2d$y const y
\end{lstlisting}
Using the \texttt{?} ensures that we get the function itself, rather
than trying to run it\footnote{In reality we would probably use
the optional list parameter to \texttt{import}, as described in
Section~\ref{packages}, although this would leave the functions
open for redefinition.}.

\subsection{Forward declarations (deferred declarations)}
\index{forward declaration}
It's also possible to use this mechanism to define global variables,
use them as function names, and change their values later. This lets
us defer function definitions:
\begin{v}
global foo      # define a global called ``foo'' (null-valued)
:bar foo;       # a function which uses it

:foo...;        # actually define foo
\end{v}

\section{Plugin libraries (currently Linux only)}
\label{library}
\indw{library}\index{libraries}
These are shared libraries which can be loaded into Angort. They communicate
with Angort via a simplified interface, and are easy to write in C++ (see
the plugins directory for some examples). 
Once created, they should be named with the \texttt{.angso} extension
and placed in Angort's library search path (a colon separated string).
By default, this is 
\begin{v}
.:~/.angort:/usr/local/share/angort
\end{v}
but it can be changed using the 
\texttt{searchpath} property (note that \verb+~+ will be expanded
to the user's home directory). For example, to append \texttt{/home/foo/bin}
to the path, you could write
\indw{searchpath}
\begin{v}
?searchpath ":/home/foo/bin" + !searchpath
\end{v}
Plugin libraries are loaded using the \texttt{library} word, which
leaves a namespace ID (an \emph{nsid})\index{nsid} on the stack so that \texttt{import} can be
used, or \texttt{drop} to not import anything. The namespace is
the library name, which may not be that of the library file --- it's defined
in the plugin code. For example, to load the standard file IO library
and import it, we would write
\begin{v}
`io library import
\end{v}
Again,\textbf{Note that} unlike package loading with \texttt{require},
the library name comes first.
This is because \texttt{library} is an instruction which is compiled and
then run, rather than a compiler directive which is acted on
at compile time. This means we can do things like
\begin{v}
[`id3, `mpc, `io] each {i library import}
\end{v}
to import lists of libraries.

