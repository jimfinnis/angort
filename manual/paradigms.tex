\section{Useful paradigms}
Angort is a fairly novel language, being an update of stack-based
languages with collections and functional elements. As such, new
paradigms for performing familiar tasks must be found, and often
features of the language allow these tasks to be performed in
an efficient manner but need to be ``discovered''
by actually using the language. In this section I hope to show some
examples of a few useful paradigms I have discovered during development.


\subsection{Use of the comma operator for list building}
A trivial example
is the list comprehension, a feature whereby a new list can be constructed
by performing some operation on elements of another. In Python, for example,
one can obtain the squares of a range of elements using a comprehension thus:
\begin{lstlisting}[language=Python]
print [x**2 for x in range(20)]
\end{lstlisting}
Naturally one can use \texttt{map} for this in Angort:
\begin{lstlisting}
0 20 range (dup *) map "," intercalate.
\end{lstlisting}
but I found I had this facility even before the \texttt{map} function
had been written, simply by using
the comma outside the normal list context:
\begin{lstlisting}
[] 0 20 range each {i dup *,} "," intercalate.
\end{lstlisting}
Breaking this down:
\begin{lstlisting}
[]          # stack an empty list
0 20 range  # stack a range from 0-19 inclusive
each {      # pop and iterate over that range. Empty list is now stack top.
 i          # push the next element
 dup *      # square it
 ,          # append it to the list
}           # end loop

"," intercalate.      # print the resulting list
\end{lstlisting}
In a list context, the stack picture for the comma word is \texttt{(list item -- list)}:
it appends the item to the list and leaves the list on the stack. While this
permits the intuitive \texttt{[1,2,3,4]} syntax for list ``literals'', it
also provides a powerful list-building facility. In a hash context, the
stack picture is \texttt{(hash value key -- hash)}\footnote{Which context
is used depends on the type of second item on the stack: if it is a list,
the list context is used; otherwise the type of the third item is
checked and the hash context is used if it is a hash. If not, an exception
is thrown.}. This provides a way to generate hashes:
\begin{lstlisting}
[%] 0 20 range each {i i dup*,}
\end{lstlisting}
will generate a hash of the squares of the first 20 non-negative integers,
keyed by the integer.

\subsection{Objects with methods and private members using hashes and closures}
\index{objects}\index{delegates}
We can emulate objects in Angort by creating objects as
hashes, with methods as functions closing over the local variables
in the creating function:

\begin{lstlisting}
# create a rectangle object as a hash, with delegates to draw it
:mkrectangle |x,y,w,h:|
    [%
        # a member to draw the rectangle - this
        # will create a closure over x,y,w,h.
        # We assume there is a graphics package with
        # a drawrect function; this is not a part of standard
        # Angort (although see the SDL plugin in angortplugins).
        
        `draw (?x ?y ?w ?h graphics$drawrect),
        
        # another to move it by some amount
        
        `move (|dx,dy:| 
            ?x ?dx + !x
            ?y ?dy + !y
        )
    ]
;   

# create it
100 100 10 10 mkrectangle !R
# draw it
?R?`draw@
# move it and redraw
20 20 ?R?`move@
?R?`draw@
\end{lstlisting}
Values such as \verb+?R?`move+ are actually closer to C\# ``delegates'', in that they 
contain information on the object (actually a closure), and the method to perform.
\index{private and public members}
This also provides a ``private member'' mechanism: if a value is defined
in a closure in the function which creates the hash, rather than in
the hash itself, that value will only be visible to functions defined
in the hash.

Public member variables can be implemented by storing the hash
itself as a local variable called (say) \texttt{this}, and accessing it inside
the methods. Using this technique, the above could be written as 
\begin{lstlisting}
# create a rectangle object as a hash, with delegates to draw it
:mkrectangle |x,y,w,h:this|
    [% dup !this # store the hash itself in the closure
        # set the member values
        `x ?x,
        `y ?y,
        `w ?w,
        `h ?h,
        
        `draw (?this?`x ?this?`y 
            ?this?`w ?this?`h graphics$drawrect),
        
        `move (|dx,dy:| 
            ?this?`x ?dx + !this?`x
            ?this?`y ?dy + !this?`y
        )
    ]
;   
\end{lstlisting}

\todo{Look at how to do inheritance, particularly using members in the superclass.}




\subsection{Control languages}
{\color{red}
\begin{itemize}
\item private/public sections
\item word definition
\item prompt changing
\end{itemize}
}
