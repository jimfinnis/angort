\section{Introduction}
Angort\footnote{The name is an entirely random pair of syllables,
it has no significance.}
is a stack-based concatenative programming language with some
functional features. The language has grown from a simple Forth-like
core over time, and has been used primarily for robot control on
an ExoMars rover locomotion prototype.

This is an extremely brief introduction to the language. It may
be useful for readers unfamiliar with this style of programming
to look into Forth, which is an older, more primitive (but faster and
smaller) language from which much of the syntax of Angort was
borrowed.

It combines the power and ease of a modern dynamic language with
the convenience of a Forth-like language for controlling robots in
real time. For example, on our rover we have the following definitions
in the startup file:

\begin{lstlisting}
# define a constant "wheels" holding a range from 1-6 inclusive

range 1 7 const wheels

# define a new function "d" with a single parameter "speed"

:d [speed:]

    # set a help text for this function

    :"(speed --) set speed of all drive motors"
    
    # for each wheel, set the required speed to the value
    # of the parameter
    
    wheels each {
        ?speed i!drive
    }
;

# slightly more complex function for steering

:t [angle:]
    :"(angle --) turn front wheels one way, back wheels opposite way"
    
    ?angle dup 1!steer 2!steer
    0 0 3!steer 4!steer
    ?angle neg dup 5!steer 6!steer
;

# define a function to stop the rover by setting all speeds to zero

:s 0 d;

\end{lstlisting}
Once these words are defined we can steer the robot in real time with
commands like:
\begin{v}
2500 d
30 t
s
\end{v}
These will set the rover speed to 2500, turn it to 30 degrees, and stop
it respectively. We can also directly type things like:
\begin{v}
wheels each { i dactual .}
\end{v}
which will print the actual speeds of all the wheels.
It's also possible to perform some functional programming with
anonymous functions:
\begin{v}
:sum |list:| 0 ?list (+) reduce;
\end{v}
will allow us to sum a list and print the results:
\begin{v}
[1,2,3,4,5] sum .
\end{v}
or even:
\begin{v}
1 1001 range (dup*) map sum .
\end{v}
to print the sum of the squares of the first 1000 integers.
As can be seen, Angort is a very terse language.

In the examples given so far, functions such as \texttt{dactual},
\texttt{!drive} and \texttt{?drive} are links to native C++ code: it
is very easy to interface Angort with C++.

\subsection{Downloading and building Angort}
Angort can be downloaded from \url{https://github.com/jimfinnis/angort} .
Once downloaded, it can be built and installed 
with the following commands (from
inside the top-level Angort directory):
\begin{v}
mkdir build
cd build
cmake .. -DCMAKE_BUILD_TYPE=Release
make
sudo make install
\end{v}
This is for a Linux machine
with CMake and the \texttt{readline} development libraries. The
interpreter will then be installed, typically as \texttt{/usr/local/bin/angort}. 
A small set of Angort libraries (i.e. libraries written in Angort) will also be installed
into \texttt{/usr/local/share/angort}.

A set
of native C++ plugin libraries can also be downloaded from
\url{https://github.com/jimfinnis/angortplugins}. Once Angort has
been installed, these can be built and installed with
\begin{v}
./buildall
sudo ./install
\end{v}
They will also be installed into \texttt{/usr/local/share/angort}.
Using \texttt{reallybuildall} will attempt to build extra libraries
which require additional packages, such a a CURL interface, SDL
graphics support and JACK MIDI support. Many of these have been written for
somewhat esoteric purposes as I have needed them.


