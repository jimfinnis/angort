\clearpage\section{Writing Angort plugin libraries}
Angort plugin libraries add C++ functions to Angort. They can also
add properties (see Sec.~\ref{properties}) and binary operator
overrides.

Plugins are written in C++, but for convenience this C++ is passed
through a Perl preprocessor called \texttt{makeWords.pl}. This automatically
adds code for popping arguments, adding the functions to a special
array, and initialising the plugin in a special entry point. Let's look
at a simple "hello world" example.

\lstinputlisting[language=c++]{hello.cpp}

This code will create a plugin with a \texttt{hello} function which
takes a name and generates a string ``Hello (name), how are you?''

\subsection{Angort internals: values and types}
\label{typeinternals}
\index{type!internals}
All but the most simple plugins require some knowledge of how
Angort stores and types its values. Every value in Angort is
a \texttt{Value} object. This is a class containing a union, which
holds the actual value; and a \texttt{Type} object pointer, defining
its behaviour. Each Angort type is defined by a singleton
instance of a subclass of \texttt{Type} -- all the internal types
are global variables within the \texttt{Types} namespace.
For example, strings are defined by the \texttt{StringType} object,
which has \texttt{Types::tString} pointing to its only instance.

In the case of simple types like integers, the type object defines
accessors to simple members of the \texttt{Value} class' embedded
union \texttt{d}. Integers, for example, store their data in
\texttt{d.i} while floats use \texttt{d.f}. Garbage collected types
all use the \texttt{d.gc} pointer, which points to a
\texttt{GarbageCollected} object. Thus, lists and hash value
classes -- and
many user classes -- inherit \texttt{GarbageCollected}.

All the internal types are declared in \texttt{include/types/} and
defined in \texttt{lib/types/} should you wish to study them.
This is all quite complex at first, particularly the behaviour of
garbage collected objects. It can, however, be boiled down to a few
principles:
\begin{itemize}
\item \texttt{Values} are initialised to \texttt{none}; their union's
value is irrelevant and their type field points to \texttt{Types::tNone}.
\item Calling \texttt{clr()} on a \texttt{Value} will decrement its
reference count and set the type field to \texttt{Types::tNone}.
\item Each type object contains a method to set a value to a given
type and initialise it to a value. For example
\begin{lstlisting}[language=c++]
Value v;
Types::tInteger->set(&v,1);
\end{lstlisting}
will set the value \texttt{v} to the integer 1.
Some of these may return a new object: for example, to set a value
to a list we would call
\begin{lstlisting}[language=c++]
Value v;
ArrayList<Value> *list = Types::tList->set(&v);
\end{lstlisting}
which will create a new garbage collected list, set the value's union
to hold its pointer, set the value's type field to 
\texttt{Types::tList}, and return the new list object.
\item Each type object will also contain a \texttt{get()} method
to return the underlying C++ value of a \texttt{Value}, performing
conversion if possible. For example,
\begin{lstlisting}[language=c++]
int i = Types::tInteger->get(&v);
\end{lstlisting}
will try to get the integer value from a \texttt{Value}.
\item \texttt{Value *Angort::pushval()} will push a new value onto the Angort
stack and return it. It does this by just returning a pointer to
the stack top and incrementing that pointer.  It will \emph{not} call \texttt{clr()} on the value.
\item \texttt{Value *Angort::popval()} will pop the stack, returning a pointer
to the previous stack top.
\item There are helper \texttt{Angort::push..()} and
\texttt{Angort::pop..()} methods for most basic types.
\item There are \texttt{Value::to..()} helper methods for most
types to convert and return values. For example, \texttt{toFloat()} will
attempt to convert to, and return, a C++ float.
\item Strings are a special case -- popping a string returns a
\texttt{StringBuffer} object, whose underlying buffer can be fetched
with \texttt{get()}. This is a UTF-8 string, and \texttt{StringBuffer} 
has methods for converting to wide characters. A common idiom fetches
the string buffer as a const reference and calls \texttt{get()} to
return the underlying buffer:
\index{StringBuffer class}
\begin{lstlisting}[language=c++]
const StringBuffer& s = v->toString();
...
printf("%s\n",s.get());
\end{lstlisting}
\end{itemize}

There is a small hierarchy of classes which inherit the \texttt{Type} class:
\texttt{IntegerType} (with the singleton pointer \texttt{Types::tInteger} is
just a subclass of \texttt{Type}, while \texttt{ListType} and
\texttt{HashType} inherit \texttt{GCType}, which handles values which
are garbage-collected.

Summing up the situation for \texttt{Hash} as an example,
\begin{itemize}
\item \texttt{HashObject} inherits \texttt{GarbageCollected} and
wraps a raw \texttt{Hash}: a map from \texttt{Value*} to
\texttt{Value*}.
\item \texttt{Types::tHash} points to a singleton instance of
\texttt{HashType}, which inherits \texttt{GCType}, which inherits
\texttt{Type}.
\item This singleton has methods which can create a and
set a \texttt{Value} to a new \texttt{HashObject},
and retrieve a \texttt{Hash} from a value (if it is of the correct type, throwing
an exception otherwise). The \texttt{set()} and \texttt{get()} methods
deal entirely with the underlying \texttt{Hash}, wrapping and unwrapping
in \texttt{HashObject}, which exists solely to define an
interface for garbage collection and iteration.
\end{itemize}
While this is all very complex, there are plenty of examples of words
using these concepts in the Angort source --
particularly in \texttt{stdWords.cpp}, \texttt{stdString.cpp} and
\texttt{stdColl.cpp}.




\subsection{Building plugins}
\index{plugins}
\index{C++}
To build the plugin, it needs to be passed through \texttt{makeWords.pl} 
(which can be found in the Angort source distribution) and compiled and linked as
a shared library. The steps for the example are:
\begin{v}
perl makeWords.pl hello.cpp >hello.plugin.cpp
g++ -fPIC -c -Wall hello.plugin.cpp
g++ -shared -Wl,-soname,hello.angso hello.plugin.o
\end{v}
The result is \texttt{hello.angso} where \texttt{.angso} is the extension
required for a plugin.


\subsection{Special \% directives}
The first thing to note is that some lines start with special
directives beginning with a percent sign. These are handled by
\texttt{makeWords.pl} and generate extra code inside the \texttt{.plugin.cpp} file.

\subsubsection{The \%name directive}
This gives the name of the plugin's namespace. This should be unique
to each plugin. Functions in the plugin will have the fully qualified
name \verb+namespace$function+, so our example will generate
\verb+hello$hello+. This generates a namespace block wrapping the rest
of the output, so we should be careful how we use the C++
\texttt{namespace} directive afterwards.

\subsubsection{The \%shared directive}
This says we are making a shared library, and should always be included.
It exists because \texttt{makeWords.pl} is also used to build the functions
included in the standard Angort libraries. It adds the special entry
point required by dynamically loaded plugins to register them with
Angort.

\subsubsection{The \%init and \%shutdown directives}
These are used to define functions which the plugin calls when
it is loaded and unloaded (the latter only occurs when Angort is shut
down). 

\subsubsection{The \%wordargs and \%word directives}
These is the most important directives: they define the actual functions
in the plugin. The \texttt{makeWords.pl} script generates a name
for the function and a definition, and adds the function to the
function list for the library.

The \texttt{\%word} directive is used to define a function with no
arguments, or whose arguments you wish to pop off the stack by hand.
It has the form:
\begin{v}
%word wordname (stack picture) description
more description..
more description..
{
    c++ code
}
\end{v}

If we add a function which has no arguments to our example
and run it through \texttt{makeWords.pl}, we can see what it does.
Our function is:
\begin{lstlisting}[language=c++]
%word hello2 (-- string) a test
{
    a->pushString("hello world");
}
\end{lstlisting}
and the resulting code in \texttt{hello.plugin.cpp} is:
\begin{lstlisting}[language=c++]
static void _word__hello2(angort::Angort *a){
    a->pushString("hello world");
}
\end{lstlisting}
All plugin functions become functions which take a pointer to the
\texttt{Angort} object, which is always parameter \texttt{a}. Naturally
this means you must not call any other variable \texttt{a} in your
plugin. The above code simple pushes a string onto the stack.

The \texttt{\%wordargs} directive is used if the function has arguments,
and pops them automatically, converting the types, into local variables
called \texttt{p0}, \texttt{p1} and so on. Consider a definition

\begin{lstlisting}[language=c++]
%wordargs distance dddd (x1 y1 x2 y2 -- dist)
{
    float dx = (p0-p2);
    float dy = (p1-p3);
    float dist = sqrt(dx*dx+dy*dy);
    a->pushDouble(dist);
}
\end{lstlisting}
Here, the \texttt{dddd} indicates the types of the arguments: they
are all double-precision floating point.
If a value popped from the stack is not a double,
Angort will attempt to convert it (see Sec.~\ref{types}).
Once popped, the distance is calculated and the resulting double
is pushed. The code generated is
\begin{lstlisting}[language=c++]
static void _word__distance(angort::Angort *a)
{
Value *_parms[4];
a->popParams(_parms,"dddd");
double p0 = _parms[0]->toDouble();
double p1 = _parms[1]->toDouble();
double p2 = _parms[2]->toDouble();
double p3 = _parms[3]->toDouble();
    float dx = (p0-p2);
    float dy = (p1-p3);
    float dist = sqrt(dx*dx+dy*dy);
    a->pushDouble(dist);
}
\end{lstlisting}
It uses the internal \texttt{popParams} function to pop the parameters
into an array, typechecking them as it does so, and then creates the
local variables \texttt{p0, p1}... as discussed above, converting
to the appropriate types. Then our code follows.

Lists and hashes are rather more complex. In the example below,
we create a list and add two strings to it:
\begin{lstlisting}[language=c++]
%wordargs stringstolist ss (string string -- list) list from two strings
{
    Value v;
    ArrayList<Value> *list = Types::tList->set(&v);
    Types::tString->set(list->append(),p0);
    Types::tString->set(list->append(),p1);
    a->pushval()->copy(&v);
}
\end{lstlisting}
The two parameters \texttt{p0} and \texttt{p1} are popped and
initialised as \texttt{const char *} by the prologue created
by \texttt{makeWords.pl}. Then a local \texttt{Value} variable
is declared,
and set to hold a new list by a call to the \texttt{set()} method
of the list type object, \texttt{Types::tList} (see Sec.~\ref{typeinternals}),.
The new list is stored in the variable \texttt{list.} 
Then, for each string, a call is made to the list's \texttt{append()} method
which returns a new \texttt{Value} pointer, and the string type
object's \texttt{set()} method is used to set this new value to a string.
Finally, a new value is pushed onto the stack, and \texttt{v} -- the value
which wraps the list -- is copied into it.

\subsubsection{Important ``gotcha'': overwriting stack values}
Note that our code above doesn't look like this:
\begin{lstlisting}[language=c++]
%wordargs stringstolist ss (string string -- list) list from two strings
{
    ArrayList<Value> *list = Types::tList->set(a->pushval());
    Types::tString->set(list->append(),p0);
    Types::tString->set(list->append(),p1);
}
\end{lstlisting}
That is, we don't push a new value onto the stack and start writing
to it immediately. That's because the following sequence would occur:
\begin{itemize}
\item pop the stack into p0 and p1 (in the prologue)
\item push the stack
\item set the new stack top to be a list -- overwriting the value
which was popped into p0, replacing the string pointer with a list
pointer
\end{itemize}
This is why we write our return value into a local variable which
we push at the end of function.

\clearpage
\subsubsection{\%wordargs type specification}
The string immediately following the function name in a \texttt{\%wordargs}
specification gives the types of the parameters, each by a single
character. If you want a variadic function, use \texttt{\%word} and
the Angort \texttt{pop..()} functions to pop the values one by one.
The type letters are given below.
\begin{center}
\begin{tabular}{|l|l|l|} \hline
Letter & C++ type & notes \\ \hline
n & \texttt{float} & \\
L & \texttt{long} & \\
b & \texttt{bool} & \\
i & \texttt{integer} & \\
d & \texttt{double} &\\
c or v  & \texttt{Value *} & raw Angort value\\
s or S & \texttt{const char *} & string or symbol \\
y & \texttt{const char *} & string, symbol or none (NULL)\\
l & \texttt{ArrayList<Value>*} & list \\
h & \texttt{Hash *} & hash \\
A or B & user type (see below) & \\ \hline
\end{tabular}
\end{center}

\subsubsection{The \%type directive and user types}
\index{type!user defined in C++}
This is used to tell \texttt{makeWords.pl} about a user-defined
type. It takes three arguments: the name of the type,
the static global variable containing the type object,
and the name of the object returned by calling the type object's
\texttt{get()} method with a \texttt{Value} pointer.

To define a user type, we need to define a type object for it.
Given that most user types are rather more than simple values,
this example will cover creating new garbage-collected types.
Let's consider complex numbers, which require the storage of two
double-precision values: real and imaginary components.
Our complex number class will look like this, inheriting
\texttt{GarbageCollected}:
\begin{lstlisting}[language=c++]
class Complex : public GarbageCollected {
public:
    double r,i;
    
    virtual ~Complex(){}
    Complex(double _r,double _i){
        r = _r;
        i = _i;
    }
};
\end{lstlisting}
Now we can define the type object:
\begin{lstlisting}[language=c++]
class ComplexType : public GCType {
public:
    ComplexType(){
        add("complex","CPLT");
    }
    
    virtual ~ComplexType(){}
    
    Complex *get(const Value *v) const {
        if(v->t!=this)
            throw RUNT(EX_TYPE,"not a complex number");
        return (Complex *)(v->v.gc);
    }
    
    void set(Value *v,double r,double i){
        v->clr();
        v->t=this;
        v->v.gc = new Complex(r,i);
        incRef(v);
    }
    
    virtual const char *toString(bool *allocated,const Value *v) const {
        char buf[128];
        Complex *w = get(v);
        snprintf(buf,128,"%f+%fi",w->r,w->i);
        *allocated=true;
        return strdup(buf);
    }
};
\end{lstlisting}
This code should appear after the \texttt{\%name} and
\texttt{\%shared} directives.
The important members here are \texttt{get()}, which ensures the
passed \texttt{Value} is the right type and returns the underlying
structure; and \texttt{set()}, which clears the value of any
previous data, sets the type pointer, initialises a new object
and increments the reference count. Also of note is
the constructor, which calls \texttt{add()} to register the type
with Angort, passing in two unique identifiers -- long and 4-character
(the latter for quick comparison). Finally, we provide a string conversion
method -- this type will be automatically converted to a string
when required.

Now we have our types, we can register them:
\begin{lstlisting}[language=c++]
static ComplexType tC;
%type complex tC Complex
\end{lstlisting}
This declares and initialises the singleton type object,
and registers the type with \texttt{makeWords.pl} under the name
\texttt{complex}.
We can now define words which use this type. First, the easy case
of making a new complex number:
\begin{lstlisting}[language=c++]
%wordargs complex dd (r i -- complex)
{
    tC.set(a->pushval(),p0,p1);
}
\end{lstlisting}
This just pushes a value, and passes it to our new type object's
\texttt{set()} method with the parameters (two doubles).

If we want to write a word which takes a complex number as a parameter,
we need to use the user type facility of \texttt{\%wordargs}. If you
specify A or B as a type, a user type name or two (comma-separated) should
follow the specification after a vertical bar. This is the type
name as given as the first argument of \texttt{\%type}. Here are
functions to extract the real and imaginary components:
\begin{lstlisting}[language=c++]
%wordargs real A|complex (complex -- real part)
{
    a->pushDouble(p0->r);
}

%wordargs img A|complex (complex -- img part)
{
    a->pushDouble(p0->i);
}
\end{lstlisting}

\clearpage
\subsubsection{Defining binary operators}
We can define functions for the binary operators by using
\texttt{\%binop} and the names of two types, separated by the operator
name, one of: \texttt{equals}, \texttt{nequals}, \texttt{add}, \texttt{mul},
\texttt{div}, \texttt{sub}, \texttt{and}, \texttt{or}, \texttt{gt}, 
\texttt{lt}, \texttt{le}, \texttt{lt}, \texttt{mod}, \texttt{cmp}. 
The types can be user-defined or internal:
\begin{lstlisting}[language=c++]
%binop complex mul complex
{
    Complex *l = tC.get(lhs); // use tC to get Complex pointer from Value
    Complex *r = tC.get(rhs);
    // calculate result
    double real = (l->r * r->r) - (l->i * r->i);
    double img = (l->r * r->i) + (l->i * r->r);
    // push new complex.
    tC.set(a->pushval(),real,img);
}

%binop complex add complex
{
    Complex *l = tC.get(lhs);
    Complex *r = tC.get(rhs);
    tC.set(a->pushval(),l->r+r->r,l->i+r->i);
}

%binop complex add double
{
    Complex *l = tC.get(lhs);
    float r = rhs->toDouble();
    tC.set(a->pushval(),l->r+r,l->i);
}
// there would be more down here!
\end{lstlisting}
Note that binop functions take two \texttt{Value} pointers; no
automatic parameter unwrapping \emph{\'a la} \texttt{\%wordargs} is
done. Also note that the type handling in binops is quite strict:
with the above definitions
\begin{v}
4 5 complex 1.0d + .
\end{v}
will work, but
\begin{v}
4 5 complex 1 + .
\end{v}
will not, because there is no registered binary operator which
takes a complex number and an integer.
